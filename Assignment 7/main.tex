%%%%%%%%%%%%%%%%%%%%%%%%%%%%%%%%%%%%%%%%%%%%%%%%%%%%%%%%%%%%%%%
%
% Welcome to Overleaf --- just edit your LaTeX on the left,
% and we'll compile it for you on the right. If you open the
% 'Share' menu, you can invite other users to edit at the same
% time. See www.overleaf.com/learn for more info. Enjoy!
%
%%%%%%%%%%%%%%%%%%%%%%%%%%%%%%%%%%%%%%%%%%%%%%%%%%%%%%%%%%%%%%%

% Inbuilt themes in beamer
\documentclass{beamer}

% Theme choice:
\usetheme{CambridgeUS}
\usepackage{amsmath}
\providecommand{\pr}[1]{\ensuremath{\Pr\left(#1\right)}}
\providecommand{\cdf}[2]{\ensuremath{\text{F}_{#1}\left(#2\right)}}

% Title page details: 
\title{Assignment 6 } 
\author{MD GUFRAN ASHRAF (BT21BTECH11003)}
\date{\today}

\begin{document}

% Title page frame
\begin{frame}
    \titlepage 
\end{frame}

% Outline frame
\begin{frame}{Outline}
    \tableofcontents
\end{frame}


% Lists frame
\section{Problem}
\begin{frame}{Problem Statement}

\textbf{(Papoulis chap-12 - 12.1 )}
Show that if a process is normal and distribution-ergodic as in (12-35), then it is also
mean-ergodic.\\
Defination from 12-35
\begin{equation*}
    \frac{1}{T}\int_{0}^{T}F(x,x:\tau)d\tau  \longrightarrow F^2(x)  
\end{equation*}
for T  \rightarrow \infty
\end{frame}


% Blocks frame
\section{Solution}
\begin{frame}{Solution}
The process x(t) is normal and such that
\begin{equation}
    F(x,x:\tau) \longrightarrow F^2(x) 
\end{equation}
as \tau \rightarrow \infty\\
We shall show that it is mean - ergodic. It suffices to show that
\begin{equation*}
    C(\tau) \rightarrow 0 \\ for \tau \rightarrow \infty
\end{equation*}
We are assuming that \eta=0 \\  C(0)=1
\end{frame} 

\begin{frame}

\begin{equation}
    f(x_{1},x_{2};\tau) = \frac{1}{2\pi\sqrt{1-r^2}}exp\left|-\frac{1}{2(1-r^2)}(x^2_{1}-2rx_{1}x_{2}+x^2_{2})\right|
\end{equation}
\begin{equation}
   = \frac{1}{2\pi\sqrt{1-r^2}}exp\left|-\frac{1}{2(1-r^2)}(x_{1}-rx_{2})^2\right|e^{\frac{-x^2_{2}}{2}}
\end{equation}
Clearly, f(x,y) = f(y,x) , hence
\begin{equation*}
    F(x+dx, x+dx;\tau) - F(x,x,\tau) = 2\int_{-\infty}^{x}f(\epsilon,x)d{\epsilon}dx
\end{equation*}
\begin{equation}
    =\frac{1}{\pi\sqrt{1-r^2}}\int_{-\infty}^{x}exp\left|-\frac{1}{2(1-r^2)}(\epsilon-rx)^2\right|d{\epsilon}e^{-x^2_{2}/2}dx
\end{equation}
\end{frame}
\begin{frame}
    further
    \begin{equation*}
        F^2(x+dx) - F^2(x) = 2F(x)f(x)dx
    \end{equation*}
    From above 1 it follows that
    G\left\{ \frac{x-rx}{\sqrt{1-r^2}} \right\} \longrightarrow G(x)\\  for \tau \rightarrow \infty\\ Hence, r(\tau) \rightarrow 0 \\
    as \tau \rightarrow \infty
\end{frame}



\end{document}

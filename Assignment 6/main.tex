%%%%%%%%%%%%%%%%%%%%%%%%%%%%%%%%%%%%%%%%%%%%%%%%%%%%%%%%%%%%%%%
%
% Welcome to Overleaf --- just edit your LaTeX on the left,
% and we'll compile it for you on the right. If you open the
% 'Share' menu, you can invite other users to edit at the same
% time. See www.overleaf.com/learn for more info. Enjoy!
%
%%%%%%%%%%%%%%%%%%%%%%%%%%%%%%%%%%%%%%%%%%%%%%%%%%%%%%%%%%%%%%%

% Inbuilt themes in beamer
\documentclass{beamer}

% Theme choice:
\usetheme{CambridgeUS}
\usepackage{amsmath}
\providecommand{\pr}[1]{\ensuremath{\Pr\left(#1\right)}}
\providecommand{\cdf}[2]{\ensuremath{\text{F}_{#1}\left(#2\right)}}

% Title page details: 
\title{Assignment 6 } 
\author{MD GUFRAN ASHRAF (BT21BTECH11003)}
\date{\today}

\begin{document}

% Title page frame
\begin{frame}
    \titlepage 
\end{frame}

% Outline frame
\begin{frame}{Outline}
    \tableofcontents
\end{frame}


% Lists frame
\section{Problem}
\begin{frame}{Problem Statement}

\textbf{(Papoulis chap-12 - 12.1 )}
Find the mean and variance of the random variable
\begin{equation*}
    n_{T} = \dfrac{1}{2T}\int_{-T}^{T}x(t)dt
\end{equation*}
\begin{equation*}
  where x(t)=10+v(t)  
\end{equation*}
For T=5 and for T=100, assume
\begin{equation*}
    E[v(t)] = 0 
\end{equation*}
        \begin{equation*}
            R_v(\tau) = 2\delta(\tau)
        \end{equation*}                          

\end{frame}


% Blocks frame
\section{Solution}
\begin{frame}{Solution}
\begin{equation}
    x(t)=10+v(t) 
\end{equation}
\begin{equation}
      R_v(\tau) = 2\delta(\tau)
\end{equation}
From question
\begin{equation}
     E[v(t)] = 0
\end{equation}
\begin{equation}
    E(x(t)) = E( 10 + v(t) )
    \end{equation}
    
\begin{equation}
    E[n_{T}] = E[x_{T}] = 10
\end{equation}
\begin{equation}
        C_{x}(\tau) = 2\delta(\tau)
\end{equation}
\end{frame} 

\begin{frame}
\begin{equation*}
    \sigma^2_{n_{T}} = \frac{1}{2T}\int_{-T}^{T}C_{x}(\tau)(1 - \frac{\tau}{2T})dt
\end{equation*}
From equation 6
\begin{equation*}
    \frac{1}{2T}\int_{-T}^{T}C_{x}(\tau)(1 - \frac{\tau}{2T})dt = \frac{1}{2T}\int_{-T}^{T}2\delta(\tau)(1 - \frac{\tau}{2T})dt  
\end{equation*}
\begin{equation*}
    \frac{1}{2T}\int_{-T}^{T}2\delta(\tau)(1 - \frac{\tau}{2T})dt = \frac{1}{T}
\end{equation*}
So variance \sigma^2 is $\dfrac{1}{T}$  
\end{frame}




\end{document}

%%%%%%%%%%%%%%%%%%%%%%%%%%%%%%%%%%%%%%%%%%%%%%%%%%%%%%%%%%%%%%%
%
% Welcome to Overleaf --- just edit your LaTeX on the left,
% and we'll compile it for you on the right. If you open the
% 'Share' menu, you can invite other users to edit at the same
% time. See www.overleaf.com/learn for more info. Enjoy!
%
%%%%%%%%%%%%%%%%%%%%%%%%%%%%%%%%%%%%%%%%%%%%%%%%%%%%%%%%%%%%%%%

% Inbuilt themes in beamer
\documentclass{beamer}

% Theme choice:
\usetheme{CambridgeUS}
\usepackage{amsmath}
\providecommand{\pr}[1]{\ensuremath{\Pr\left(#1\right)}}
\providecommand{\cdf}[2]{\ensuremath{\text{F}_{#1}\left(#2\right)}}

% Title page details: 
\title{Assignment 4 } 
\author{MD GUFRAN ASHRAF (BT21BTECH11003)}
\date{\today}

\begin{document}

% Title page frame
\begin{frame}
    \titlepage 
\end{frame}

% Outline frame
\begin{frame}{Outline}
    \tableofcontents
\end{frame}


% Lists frame
\section{Problem}
\begin{frame}{Problem Statement}

\textbf{(Papoulis chap-8 - 8.17 )}
Suppose that the IQ scores of children in a certain grade are the samples of an N({\eta,\sigma}).\\We test 10 children and obtain the following averages: x = 90, s = 5.
Find the 0.95 confidence interval of \sigma  and  \eta

\end{frame}


% Blocks frame
\section{Solution}
\begin{frame}{Solution}
We have n=10 so,
\begin{equation}
   n-1=10-1=9
\end{equation}
We have one tail condition
So, from t-table we can say for 95\% confidence and n=10 we have

    $$ t_{0.95}(10-1)=t_{0.95}(9)=2.26 $$
As we know
\begin{equation}
    \eta=\Bar{x} 
\end{equation}
so 95\% of \eta \\will lie between
\end{frame} 

\begin{frame}

\begin{equation}
   \Bar{x}-\frac{ts}{\sqrt{n}} < \eta < \Bar{x} + \frac{ts}{\sqrt{n}}
\end{equation}
\begin{equation}
   90-\frac{2.26*5}{\sqrt{10}} < \eta < 90 + \frac{2.26*5}{\sqrt{10}}
\end{equation}

\begin{equation}
    86.43 < \eta < 93.57
\end{equation}
For one tail level of confidence(\alpha)\\ 
\begin{equation}
    \alpha=0.025
\end{equation}
\end{frame}

\begin{frame}
Chi squared value value are as follows from chi table
\begin{equation}
    \chi_{L}^2=\chi_{1-\alpha}^2=\chi_{0.975}^2(9)=19.02
\end{equation}
\begin{equation}
    \chi_{R}^2=\chi_{\alpha}^2=\chi_{0.025}^2(9)=2.70
\end{equation}
Now,
\begin{equation}
    \frac{(n-1)s^2}{\chi_{L}^2} < \sigma^2 < \frac{(n-1)s^2}{\chi_{R}^2}
\end{equation}
\begin{equation}
    \frac{9(5^2)}{19.02} < \sigma^2 < \frac{9(5^2)}{2.7}
\end{equation}
So
\begin{equation}
     3.44 < \sigma < 9.13
\end{equation}

\end{frame}



\end{document}

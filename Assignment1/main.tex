%%%%%%%%%%%%%%%%%%%%%%%%%%%%%%%%%%%%%%%%%%%%%%%%%%%%%%%%%%%%%%%
%
% Welcome to Overleaf --- just edit your LaTeX on the left,
% and we'll compile it for you on the right. If you open the
% 'Share' menu, you can invite other users to edit at the same
% time. See www.overleaf.com/learn for more info. Enjoy!
%
%%%%%%%%%%%%%%%%%%%%%%%%%%%%%%%%%%%%%%%%%%%%%%%%%%%%%%%%%%%%%%%


% Inbuilt themes in beamer
\documentclass{beamer}

% Theme choice:
\usetheme{CambridgeUS}

\setbeamertemplate{caption}[numbered]{}

\usepackage{enumitem}
\usepackage{tfrupee}
\usepackage{amsmath}
\usepackage{amssymb}
\usepackage{gensymb}
\usepackage{graphicx}
\usepackage{txfonts}

\def\inputGnumericTable{}

\usepackage[latin1]{inputenc}                                
\usepackage{color}                                            
\usepackage{array}                                            
\usepackage{longtable}                                        
\usepackage{calc}                                            
\usepackage{multirow}                                        
\usepackage{hhline}                                          
\usepackage{ifthen}
\usepackage{caption}
\captionsetup[table]{skip=3pt}  
\providecommand{\pr}[1]{\ensuremath{\Pr\left(#1\right)}}
\providecommand{\cbrak}[1]{\ensuremath{\left\{#1\right\}}}
\renewcommand{\thefigure}{\arabic{table}}
\renewcommand{\thetable}{\arabic{table}}

% Title page details:
\title{AI1110: Assignment 1}
\author{Md Gufran Ashraf\\BT21BTECH11003}
\date{\today}
%\logo{\large \LaTeX{}}

\providecommand{\pr}[1]{\ensuremath{\Pr\left(#1\right)}}

\begin{document}

% Title page frame
\begin{frame}
    \titlepage
\end{frame}

% Remove logo from the next slides
%\logo{}


% Outline frame
\begin{frame}{Outline}
    \tableofcontents
\end{frame}


% Lists frame
\section{Question}
\begin{frame}{Question}

\textbf{Q10 - a)} : 
The marks obtained by $120$ students in an English test are given below:\\
\bigskip

\begin{tabular}{|p{0.9cm}|p{0.3cm}|p{0.3cm}|p{0.3cm}|p{0.3cm}|p{0.3cm}|p{0.3cm}|p{0.3cm}|p{0.3cm}|p{0.3cm}|p{0.4cm}|}
\hline
   Marks & $0-10 & 10-20 & 20-30 & 30-40 & 40-50 & 50-60 & 60-70 & 70-80 & 80-90 & 90-100 \\ \hline
   No. of students  & 5 & 9 & 16 & 22 & 26 & 18 & 11 & 6 & 4 & 3 $ \\ \hline
   
\end{tabular}

\bigskip

Estimate:\\
(i) the median marks.\\
(ii) the number of students who did not pass the test if the pass percentage was 50\\ (iii) the upper quartile marks.\\

\end{frame}

\section{Solution}
\begin{frame}{Solution}

\begin{tabular}{|p{1cm}|p{1cm}|p{1cm}|p{1cm}|}
\hline
     C.I & Marks less than & No. of students &  C.F.\\ \hline
   $ 0-10  & 10 & 5 & 5 \\ \hline
    10-20 & 20 & 9 & 14 \\ \hline
    20-30 & 30 & 16 & 30 \\ \hline
    30-40 & 40 & 22 & 52 \\ \hline
    40-50 & 50 & 26 & 78 \\ \hline
    50-60 & 60 & 18 & 96 \\ \hline
    60-70 & 70 & 11 & 107 \\ \hline
    70-80 & 80 & 6 & 113 \\ \hline
    80-90 & 90 & 4 & 117 \\ \hline
    90-100 & 100 & 3 & 120 $\\ \hline
  
\end{tabular}\\

\bigskip
\end{frame}

\section{Solution (Contd..)}
\begin{frame}{Solution (Contd..)}
\textbf{} No. of students = $120$\\
Median = $60$th term through cumulative marks, draw a line parellel to x-axis and then draw perpendicular line to x-axis from where it touches curve.\\
As per the graph:\\
\textbf{i) }\textbf{Median }= $42$\\

\textbf{ii)} Passing marks = $50$\\
Draw a line parellel to y-axis and perpendicular line to y-axis from where it touches curve.\\
\textbf{No. of students who did not pass} = $78$\\

\textbf{iii)} Upper quartile marks: $\frac{3}{4}$ X $n^{th} term \\
  = \dfrac{3}{4} X 120 = 90^{th}$ term\\
Draw a line parellel to x-axis and perpendicular line to x-axis from where it touches curve.\\
\textbf{Upper quartile marks}: $57$



\end{frame}
\end{document}

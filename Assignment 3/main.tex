%%%%%%%%%%%%%%%%%%%%%%%%%%%%%%%%%%%%%%%%%%%%%%%%%%%%%%%%%%%%%%%
%
% Welcome to Overleaf --- just edit your LaTeX on the left,
% and we'll compile it for you on the right. If you open the
% 'Share' menu, you can invite other users to edit at the same
% time. See www.overleaf.com/learn for more info. Enjoy!
%
%%%%%%%%%%%%%%%%%%%%%%%%%%%%%%%%%%%%%%%%%%%%%%%%%%%%%%%%%%%%%%%

% Inbuilt themes in beamer
\documentclass{beamer}

% Theme choice:
\usetheme{CambridgeUS}
\usepackage{amsmath}
\providecommand{\pr}[1]{\ensuremath{\Pr\left(#1\right)}}
\providecommand{\cdf}[2]{\ensuremath{\text{F}_{#1}\left(#2\right)}}

% Title page details: 
\title{Assignment3} 
\author{MD GUFRAN ASHRAF (BT21BTECH11003)}
\date{\today}

\begin{document}

% Title page frame
\begin{frame}
    \titlepage 
\end{frame}

% Outline frame
\begin{frame}{Outline}
    \tableofcontents
\end{frame}


% Lists frame
\section{Problem}
\begin{frame}{Problem Statement}

\textbf{(Papoulis chap-7 - 7.10 )}
We denote by $x_{m}$ a random variable equal to the number of tosses of a coin until heads shows
for the mth time. Show that if P[h] = P. then E_{{x}_m} = m/p

\end{frame}


% Blocks frame
\section{Solution}
\begin{frame}{Solution}
As we know 
\begin{equation}
    1+x+x^2......x^n=\frac{1}{1-x}
\end{equation}
for

    $$ -1 < x < 1 $$
Differentiating on both side with respect to x we get
\begin{equation}
    1+2x+3x^2+4x^3....(n-1)x^n..=\frac{1}{(1-x)^2}
\end{equation}
\end{frame} 

\begin{frame}
The random variable x1 equals the number of tosses until head shows for the time, \\
Hence, x1 takes the value 1,2..... with P(x1=k)=pq^(k-1).\\ Hence,
\begin{equation}
   E(x_{1})=\sum_{k=1}^{\infty}kP(x1=k)
\end{equation}
\begin{equation}
    E(x_{1})=\sum_{k=1}^{\infty}kpq^(k-1)
\end{equation}
from equation 1 we can say
\begin{equation}
    E(x_{1})=\frac{p}{(1-q)^2}
\end{equation}
as we know that p+q=1
\begin{equation}
    E(x1)=\frac{1}{p}
\end{equation}
\end{frame}

\begin{frame}
Starting the count after the first head shows , we conclude that the random variable x2-x1 has the same statistics to x1\\ Hence,
\begin{equation}
    E(X_{2}-X_{1})=E(x_{1})
\end{equation}
\begin{equation}
    E(x_{2})=2E(x_{1})=\frac{2}{p}
\end{equation}
From induction
\begin{equation}
    E(X_{n}-X_{(n-1)})=E(x_{1})
\end{equation}
\begin{equation}
    E(x_{n})=E_{X_{(n-1)}} +E(x_{1})=\frac{n-1}{p} + \frac{1}{p}=\frac{n}{p}
\end{equation}
So
\begin{equation}
     E(x_{n})=\frac{n}{p}
\end{equation}

\end{frame}



\end{document}

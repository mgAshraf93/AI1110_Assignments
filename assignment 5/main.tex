%%%%%%%%%%%%%%%%%%%%%%%%%%%%%%%%%%%%%%%%%%%%%%%%%%%%%%%%%%%%%%%
%
% Welcome to Overleaf --- just edit your LaTeX on the left,
% and we'll compile it for you on the right. If you open the
% 'Share' menu, you can invite other users to edit at the same
% time. See www.overleaf.com/learn for more info. Enjoy!
%
%%%%%%%%%%%%%%%%%%%%%%%%%%%%%%%%%%%%%%%%%%%%%%%%%%%%%%%%%%%%%%%

% Inbuilt themes in beamer
\documentclass{beamer}

% Theme choice:
\usetheme{CambridgeUS}
\usepackage{amsmath}
\providecommand{\pr}[1]{\ensuremath{\Pr\left(#1\right)}}
\providecommand{\cdf}[2]{\ensuremath{\text{F}_{#1}\left(#2\right)}}

% Title page details: 
\title{Assignment 5 } 
\author{MD GUFRAN ASHRAF (BT21BTECH11003)}
\date{\today}

\begin{document}

% Title page frame
\begin{frame}
    \titlepage 
\end{frame}

% Outline frame
\begin{frame}{Outline}
    \tableofcontents
\end{frame}


% Lists frame
\section{Problem}
\begin{frame}{Problem Statement}

\textbf{(Papoulis chap-8 - 8.41 )}
Let T(x) represent an unbiased estimator for the unknown parameter ${\psi(\theta)}$ based on the
random variables $(X_{1}, X_{2} •••• x_{n}) = x$ under joint density function f(x,$\theta$). Show that the
Crammer-Rao lower bound for the parameter $\psi(\theta)$ satisfies the inequality\\
\begin{equation*}
Var{{T(x)}} \geq \dfrac{[{\psi}^{'}({\theta})]^{2}}{E \left\{ \left(\dfrac{{\partial}{\log}f(x,{\theta})}{{\partial}{\theta}}\right)^2 \right\}}
\end{equation*}

                                  

\end{frame}


% Blocks frame
\section{Solution}
\begin{frame}{Solution}
\begin{equation}
    E[T(x)] = \int_{-\infty}^{\infty}T(x)f(x;\theta)dx = \psi(\theta)
\end{equation}
So after differentiating with respect to $\theta$
we get
\begin{equation}
    \int_{-\infty}^{\infty}T(x)\dfrac{{\partial}f(x,{\theta})}{{\partial}{\theta}}dx = {\psi}^{'}({\theta} 
\end{equation}
Replacing T(x) as $\psi(\theta)$
\begin{equation}
    \int_{-\infty}^{\infty}\psi(\theta)\dfrac{{\partial}f(x,{\theta})}{{\partial}{\theta}}dx = 0
\end{equation}
\end{frame} 

\begin{frame}
On subtrating equation 2 with 3 we get
\begin{equation}
     \int_{-\infty}^{\infty}[T(x)-\psi(\theta)]\dfrac{{\partial}f(x,{\theta})}{{\partial}{\theta}}dx = {\psi}^{'}({\theta})
\end{equation}
we can say
\begin{equation}
    \dfrac{{\partial}{\log}f(x,{\theta})}{{\partial}{\theta}}f(x;{\theta}) = \dfrac{{\partial}f(x;{\theta})}{{\partial}{\theta}}
\end{equation}
\begin{equation}
     \int_{-\infty}^{\infty}[T(x)-\psi(\theta)]f(x;{\theta})\dfrac{{\partial}{\log}f(x,{\theta})}{{\partial}{\theta}}dx = {\psi}^{'}({\theta})
\end{equation}
\end{frame}

\begin{frame}
From cauchy schwarz inequality we can give
\begin{equation*}
    E[[{T(x)-{\psi}(\theta)}]^2] \geq \dfrac{[{\psi}^{'}({\theta})]^{2}}{E \left\{ \left(\dfrac{{\partial}{\log}f(x,{\theta})}{{\partial}{\theta}}\right)^2 \right\}}
\end{equation*}

\end{frame}



\end{document}
